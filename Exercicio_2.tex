\documentclass{article}

\usepackage[brazilian]{babel}
\usepackage[utf8]{inputenc}
\usepackage[T1]{fontenc}
\usepackage{fancyhdr}

\title{Exercício 2}
\author{Flávio Barros - RA016120}

\usepackage{Sweave}
\begin{document}
\input{Exercicio_2-concordance}
\maketitle
\pagestyle{fancy}
\fancyhead{Exercício 2}
\cfoot{\thepage}
\chead{Flávio Barros}
\rhead{RA:016120}

\section{Leitura dos Dados}

Os dados do problema são imagens do tipo PGM com $64$ x $64$ pixels por imagem, onde cada pixel tem valor 1 ou 0. Cada imagem tem um nome no formato X\_ yyy.BMP.inv.pgm, onde o X é o dígito desenhado na imagem.

Assim a primeira parte do Exercício 2 é efetuar a leitura dos dados. Para isso me utilizarei do pacote \texttt{pixmap} com o qual é possível ler e manipular imagens PGM.

\begin{Schunk}
\begin{Sinput}
> library(pixmap)
\end{Sinput}
\end{Schunk}

Após isso será feita a leitura dos arquivos no conjunto de treino e do conjunto de teste. Esta consiste na criação de dois \texttt{data.frames} chamados treino e teste, e também da criação de um vetor do tipo \texttt{factor} para armazenar as classes, obtidas dos nomes dos arquivos de cada imagem.

Inicialmente é definido o local onde estão os arquivos.

\begin{Schunk}
\begin{Sinput}
> path_treino <- '/home/ra016120/Dropbox/MO444/Exercicio2/treino/'
> setwd(path_treino)